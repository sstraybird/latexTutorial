% 导言区
\documentclass[12pt]{article}

\usepackage{ctex}

\newcommand{\myfont}{\textit{\textbf{\textsf{Fancy Text}}}}
% 正文区
\begin{document}
    %字体族设置(罗马字体、无衬线字体、打字机字体)
    \textrm{Roman Family}
    \textsf{Sans Serif Family}
    \texttt{Typewriter Family}

    {\rmfamily Roman Family} {\sffamily Sans Serif Family} {\ttfamily Typewriter Family}

    \sffamily who you are? you find self on everyone around.take you as the same as others!

    \ttfamily Are you wise than others? definitely no. in some ways,may it is true.what can you achieve? a luxurious house? a brillilant car? an admirable career? who knows

    % 字体系列设置(粗细、宽度)
    \textmd{Medium Series}
    \textbf{Boldface Series}

    {\mdseries Medium Series} {\bfseries Boldface Series}

    %字体形状(直立、斜体、伪斜体、小型大写)
    \textup{Upright Shape}
    \textit{Italic Shape}
    \textsl{Slanted Shape}
    \textsc{Small Caps Shape}

    {\upshape Upright Shape} {\itshape Italic Shape} {\slshape Slanted Shape} {\scshape Small Caps Shape}

    % 中文字体
    {\songti 宋体} \quad {\heiti 黑体} \quad {\fangsong 仿宋}    {\kaishu 楷书}

    中文字体的\textbf{粗体}与\textit{斜体}
    
    %字体大小
    {\tiny Hello}	\\
    {\scriptsize Hello} \\
    {\footnotesize Hello}   \\
    {\small Hello}  \\
    {\normalsize Hello} \\  %与normalsize相对的大小,由文档类的参数控制的
    {\large Hello}  \\
    {\Large Hello}  \\
    {\LARGE hello}  \\
    {\huge hello}   \\
    {\Huge hello}   \\

    % 中文字号设置命令
    \zihao{-0} 你好!    \\
    \zihao{5} 你好!    

    % latex的思想:格式与内容分离,不建议在文档中使用大量命令,建议使用newcommand定义一个新的命令,以执行相关的操作
    \myfont
\end{document}