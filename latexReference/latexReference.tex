\documentclass{ctexart}
\usepackage[round]{natbib}
% \bibliographystyle{plain}
\bibliographystyle{plainnat}
\begin{document}

% 一次管理,一次使用
% 参考文献格式:
% \begin{thebibliography}{编号样本}
%     \bibitem[记号]{引用标志}文献条目1
%     \bibitem[记号]{引用标志}文献条目2
%     ...
% \end{thebibliography}
%   其中文献条目包括: 作者,题目,出版社,年代,版本,页码等
%   引用时候要可以采用 \cite{引用标志1,引用标志2,...}
% 引用一篇文章 \cite{article1}    引用一本书\cite{book1}等等
%     \begin{thebibliography}{99}
%         \bibitem{article1}陈立辉,苏伟,蔡川,陈小云.\emph{基于Latex的Web数学公式提取方法研究}[J].
%         \bibitem{book1} William H.Press,Saul A.Teukilsky,\emph{Numerical Recipes 3rd Edition:The Art of Sceintific Computing}
%         Cambridege University Press,New York,2007.
%         \bibitem{latexGuide} Kopka Helmut,W. Daly Patrick,\emph{Guide to \LaTeX},$4^{th}$ Edition.Availabe at \texttt{http://www.amazon.com}
%         \bibitem{latexMath} Graetzer George,\emph{Math Into \LaTeX},Birkh Boston;3 edition (June 22,2000)
%     \end{thebibliography}


这是一个参考文献的引用\cite{hufflen2006names} 
好好\cite{mittelbach2004} 
    
    \nocite{*}
    
    aaa
    
    \bibliography{test,cnki}

\end{document}